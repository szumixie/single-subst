% easychair.tex,v 3.5 2017/03/15

\documentclass{easychair}
%\documentclass[EPiC]{easychair}
%\documentclass[EPiCempty]{easychair}
%\documentclass[debug]{easychair}
%\documentclass[verbose]{easychair}
%\documentclass[notimes]{easychair}
%\documentclass[withtimes]{easychair}
%\documentclass[a4paper]{easychair}
%\documentclass[letterpaper]{easychair}

\usepackage{doc}
\usepackage{proof}
\usepackage{amssymb}
\usepackage{amsmath}
\usepackage{newtx}

% use this if you have a long article and want to create an index
% \usepackage{makeidx}

% In order to save space or manage large tables or figures in a
% landcape-like text, you can use the rotating and pdflscape
% packages. Uncomment the desired from the below.
%
% \usepackage{rotating}
% \usepackage{pdflscape}

%% Front Matter
%%
% Regular title as in the article class.
%
\title{Type theory in type theory using single substitutions}

% Authors are joined by \and. Their affiliations are given by \inst, which indexes
% into the list defined using \institute
%
\author{
Ambrus Kaposi
\and
Szumi Xie
}

% Institutes for affiliations are also joined by \and,
\institute{
   Eötvös Loránd University,
   Budapest, Hungary, 
   \email{\{akaposi|szumi\}@inf.elte.hu}
}

%  \authorrunning{} has to be set for the shorter version of the authors' names;
% otherwise a warning will be rendered in the running heads. When processed by
% EasyChair, this command is mandatory: a document without \authorrunning
% will be rejected by EasyChair

\authorrunning{Kaposi, Xie}

% \titlerunning{} has to be set to either the main title or its shorter
% version for the running heads. When processed by
% EasyChair, this command is mandatory: a document without \titlerunning
% will be rejected by EasyChair
\titlerunning{Type theory in type theory using single substitutions}

\begin{document}

\maketitle

\newcommand{\Ra}{\Rightarrow}
\newcommand{\ra}{\rightarrow}
\newcommand{\Con}{\mathsf{Con}}
\newcommand{\Var}{\mathsf{Var}}
\newcommand{\var}{\mathsf{var}}
\newcommand{\Tm}{\mathsf{Tm}}
\newcommand{\Ty}{\mathsf{Ty}}
\newcommand{\Sub}{\mathsf{Sub}}
\newcommand{\U}{\mathsf{U}}
\newcommand{\El}{\mathsf{El}}
\renewcommand{\c}{\mathsf{c}}
\newcommand{\lam}{\mathsf{lam}}
\newcommand{\app}{\mathsf{app}}
\newcommand{\oldapp}{\mathbin{\cdot}}
\renewcommand{\tt}{\mathsf{tt}}
\newcommand{\Br}{\mathsf{Br}}
\newcommand{\aps}{\mathsf{aps}}
\newcommand{\ap}{\mathsf{ap}}
\newcommand{\mkBrPi}{\mathsf{mkBr}{\Pi}}
\newcommand{\Gel}{\mathsf{Gel}}
\newcommand{\gel}{\mathsf{gel}}
\newcommand{\ungel}{\mathsf{ungel}}
\renewcommand{\sp}{\hspace{1.5em}}
\newcommand{\blank}{\mathord{\hspace{1pt}\text{--}\hspace{1pt}}} %from the book
\newcommand{\Set}{\mathsf{Set}}
\newcommand{\ext}{\mathop{\triangleright}}
\newcommand{\R}{\mathsf{R}}
\newcommand{\p}{\mathsf{p}}
\renewcommand{\S}{\mathsf{S}}
\newcommand{\vz}{\mathsf{vz}}
\newcommand{\vs}{\mathsf{vs}}

What is the syntax of Martin-Löf type theory? In this abstract, we
give a minimalistic answer to this question. We aim to define the
syntax of type theory only using operations that are unavoidable. We
would also like to eschew boilerplate by only defining well-typed terms
which are quotiented by conversion
\cite{DBLP:conf/popl/AltenkirchK16}. This means that type theory is a
generalised algebraic theory (GAT), its syntax is a quotient
inductive-inductive type (QIIT). Usual such definitions of type theory are
based on categories: category with families \cite{DBLP:journals/corr/abs-1904-00827}, locally cartesian closed
category \cite{Seely1984-SEELCC,DBLP:journals/mscs/ClairambaultD14}, contextual category \cite{DBLP:journals/apal/Cartmell86,initiality-agda}, C systems \cite{AHRENS_EMMENEGGER_NORTH_RIJKE_2023}, etc. In this abstract we
show that categories are not necessary for defining the GAT of type
theory. A category-free definition of type theory is B-systems \cite{AHRENS_EMMENEGGER_NORTH_RIJKE_2023} which
includes telescopes with complex operations and equations. In our
definition we avoid these as well. In this abstract we give a
tutorial-style introduction to our minimalistic definition of type
theory, no prior knowledge of the metatheory of type theory is
required. We introduce the operations in a naive, logical order.

We need variables in our language, so we introduce sorts of contexts,
types (which depend on a context) and variables (which are in a context and have a type).
\[
 \Con : \Set \hspace{6em} \Ty : \Con\ra\Set \hspace{6em} \Var : (\Gamma:\Con)\ra\Ty\,\Gamma\ra\Set  \\
\]
Contexts are either empty or are built from a context extended with a
type.
\[
\diamond : \Con \hspace{6em} \blank\ext\blank : (\Gamma:\Con)\ra\Ty\,\Gamma\ra\Con
\]
We define variables as well-typed De Bruijn indices, but to express
these we need to weaken types: e.g.\ the zero De Bruijn index $\vz$
has a weakened type. We introduce a new sort for substitutions $\Sub$,
an instantiation operation $\blank[\blank]$ on types, and a weakening
substitution $\p$. For now, a separate sort $\Sub$ seems like an overkill because we
are only using $\blank[\p]$, but it will come handy soon.
\[
\Sub: \Con\ra\Con\ra\Set \hspace{3em} \blank[\blank] : \Ty\,\Gamma\ra\Sub\,\Delta\,\Gamma\ra\Ty\,\Delta \hspace{3em} \p : \Sub\,(\Gamma\ext A)\,\Gamma \hspace{3em}
\]
\[
\vz : \Var\,(\Gamma\ext A)\,(A[\p]) \hspace{3em} \vs : \Var\,\Gamma\,A\ra\Var\,(\Gamma\ext B)\,(A[\p])
\]
Now we introduce $\Pi$ types together with an equation on how
instantiation with $\p$ acts on them. This is tricky: as $\Pi$ binds a
new variable in its second argument, we need a new version of the
weakening substitution which leaves the last variable untouched. This
is why we introduce lifting of a substitution $\blank^{+}$, and now we
can state a general instantiation rule for $\Pi$ which works not only
for $\p$, but arbitrary substitutions (including lifted ones).
\begin{alignat*}{10}
& \Pi:(A:\Ty\,\Gamma)\ra\Ty\,(\Gamma\ext A)\ra\Ty\,\Gamma \hspace{3em} && \blank^+ : (\gamma:\Sub\,\Delta\,\Gamma)\ra\Sub\,(\Delta\ext A[\gamma])\,(\Gamma\ext A) \\
& \Pi[] : (\Pi\,A\,B)[\gamma] = \Pi\,(A[\gamma])\,(B[\gamma^+])
\end{alignat*}
In addition to having variables, we need a sort of terms which
includes variables and lambda abstraction.
\[
\Tm : (\Gamma:\Con)\ra\Ty\,\Gamma\ra\Set \hspace{1.8em} \var : \Var\,\Gamma\,A\ra\Tm\,\Gamma\,A \hspace{1.8em} \lam : \Tm\,(\Gamma\ext A)\,B\ra\Tm\,\Gamma\,(\Pi\,A\,B)  \\
\]
To express application, we need single substitutions as well because
the argument of the function appears in the return type. In addition
to $\p$ and $\blank^+$, $\langle\blank\rangle$ is the third and last
operation for creating substitutions.
\[
\langle\blank\rangle : \Tm\,\Gamma\,A\ra\Sub\,\Gamma\,(\Gamma\ext A) \hspace{5em} \blank\cdot\blank: \Tm\,\Gamma\,(\Pi\,A\,B)\ra(a:\Tm\,\Gamma\,A)\ra\Tm\,\Gamma\,(B[\langle a\rangle])
\]
Now we would like to express the $\beta$ computation rule, but for
this we also need to be able to instantiate terms (in addition to
types).
\[
\blank[\blank] : \Tm\,\Gamma\,A\ra(\gamma:\Sub\,\Delta\,\Gamma)\ra\Tm\,\Delta\,(A[\gamma]) \hspace{3em} \Pi\beta : \lam\,t\cdot a = t[\langle a\rangle] 
\]
Now that we have instantiation of terms, we need to revisit all
operations producing terms and provide rules on how to instantiate
them: first of all, we need instantiation rules for $\lam$ and
$\blank\cdot\blank$. The rule $\lam[]$ is well-typed because of
$\Pi[]$, however ${\cdot}[]$ is not well-typed on its own and requires
a new equation $[\langle\rangle]$.
\[
\lam[] : (\lam\,t)[\gamma] = \lam\,(t[\gamma^+]) \hspace{1em} [\langle\rangle] : A[\langle a\rangle][\gamma] = A[\gamma^+][\langle a[\gamma]\rangle] \hspace{1em} {\cdot}[] : (t\cdot a)[\gamma] = (t[\gamma])\cdot(a[\gamma])
\]
Then we need instantiation rules for variables, we list these for each
possible substitution $\Sub$: weakening of a variable increases the index by
one; when instantiating with lifted substitutions and single
substitutions, we have to do case distinction on the De Bruijn index
whether it is zero or successor. For the latter two cases, we need
type equations (named $[\p][^+]$ and $[\p][\langle\rangle]$) to
typeckeck the term equations.
\begin{alignat*}{10}
  & && \var\,x[\p] = \var\,(\vs\,x) \\
  & [\p][^+] : A[\p][\gamma^+] = A[\gamma][\p] \hspace{4em}&& \var\,\vz[\gamma^+] = \var\,\vz \hspace{4em} && \var\,(\vs\,x)[\gamma^+] = \var\,x[\gamma][\p] \\
  & [\p][\langle\rangle] : A[\p][\langle a\rangle] = A && \var\,\vz[\langle a\rangle] = a &&  \var\,(\vs\,x)[\langle a\rangle] = \var\,x
\end{alignat*}
Finally, to typecheck the $\Pi\eta$ rule, we need our last equations on
types.
\[
[\p^+][\langle\vz\rangle] : A[\p^+][\langle\var\,\vz\rangle] = A \hspace{6em}\Pi\eta : t = \lam\,(t[\p]\cdot\var\,\vz)
\]
This concludes all the rules for type theory with $\Pi$ (this type theory is actually empty because there are no base types, but is enough to illustrate our method). We summarise
as follows: there are three kinds of substitutions (single weakening,
single substitution, lifted substitution), we have 5 equations
describing how instantiation acts on variables, and 4 equations which
describe general properties of instantiation on types. The rest of the
rules are specific to our single type former $\Pi$: the only extra
requirement is that each operation is equipped with an instantiation
rule ($\Pi[]$, $\lam[]$, ${\cdot}[]$). Perhaps surprisingly, this is
enough to define the syntax: there is no need for $\Con$ and $\Sub$ to
a form a category, no need for parallel substitutions, empty
substitution, parallel weakenings, telescopes, or combinations of these.

When adding new type formers, we only need the rules for the type
former, and an extra instantiation (naturality) rule for each
operation. For example, a Coquand-universe can be added by $\U :
\Ty\,\Gamma$, $\El:\Tm\,\Gamma\,\U\ra\Ty\,\Gamma$,
$\c:\Ty\,\Gamma\ra\Tm\,\Gamma\,\U$, $\U\beta : \El\,(\c\,A) = A$,
$\U\eta : \c\,(\El\,a) = a$, and three instantiation rules (note that
we would need indexing $\U$ and $\Ty$ by universe levels to avoid inconsistency). In \cite{sogat}, we showed
that any second-order generalised algebraic theory has a
single substitution presentation.

In the syntax (initial model, QIIT) of
the above theory, all the rules of categories with families (CwF
\cite{DBLP:journals/corr/abs-1904-00827}) are admissible. That is, by
induction on the single substitution syntax, we can define parallel
substitutions (lists of terms) which are composable and form a
category; we can define instantiation by parallel substitutions for
types and terms, these have the usual universal property of
comprehension. The main ingredient for this construction is the notion
of $\alpha$-normal form: a kind of normal form which is still
quotiented by $\Pi\beta$, $\Pi\eta$, but does not include explicit
substitutions. If a type is in $\alpha$-normal form, we know whether
it is $\Pi$, $\U$, or $\El$ of a term. If a term is in $\alpha$-normal
form, we know whether it is a variable, a $\lam$, an application or a
code for a type (note that any function on an $\alpha$-normal type/term has to
respect $\Pi\beta$, $\Pi\eta$, $\U\beta$, $\U\eta$). We prove
$\alpha$-normalisation (every term has a unique $\alpha$-normal form)
and then use induction on $\alpha$-normal forms to define parallel
instantiation and prove its properties.

We formalised a single substitution calculus with an infinite hierarchy
of types closed under $\Pi$ and $\U$ as a QIIT in Agda, proved
$\alpha$-normalisation, and derived all the rules for parallel
substitutions (CwF equipped with $\Pi$ types and $\U$):
\url{https://bitbucket.org/akaposi/single}. In the formalisation, we
use $\mathsf{sProp}$-valued equality and the syntax is postulated as a
QIIT with rewrite rules for its $\beta$ laws.

It is clear that the rules for the single substitution calculus are
all derivable from the CwF-rules. The other direction is however not
true: there are more models of the single substitution calculus than
the parallel substitution calculus, but the syntaxes are
isomorphic. The situation is analogous to the relationship of
combinatory logic and lambda calculus: their syntaxes are isomorphic,
but the former has more models \cite{DBLP:conf/fscd/AltenkirchKSV23}.

\label{sect:bib}
\bibliographystyle{plainurl}
%\bibliographystyle{alpha}
%\bibliographystyle{unsrt}
%\bibliographystyle{abbrv}
\bibliography{b}

%------------------------------------------------------------------------------
\end{document}

